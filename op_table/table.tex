% Opsætter KU Tex
%%%%%%%%%%%%%%%%%%%%%%%%%%%%%%%%%%%%%%%%%%%%%%%%%%%%%%%%%%%%%%%%%%%%%%%%%%%%%%%%
\documentclass[14pt]{article}
\usepackage[a4paper, hmargin={2.8cm, 2.8cm}, vmargin={2.5cm, 2.5cm}]{geometry}
\usepackage{eso-pic}  % \AddToShipoutPicture
\usepackage{graphicx} % \includegraphics
%%%%%%%%%%%%%%%%%%%%%%%%%%%%%%%%%%%%%%%%%%%%%%%%%%%%%%%%%%%%%%%%%%%%%%%%%%%%%%%%
\usepackage{float}
\usepackage{subfigure}
\usepackage{url}
\usepackage{xcolor}
\usepackage{listings}
\usepackage{caption}
% Pakker til skrifttyper, tekst osv.
%%%%%%%%%%%%%%%%%%%%%%%%%%%%%%%%%%%%%%%%%%%%%%%%%%%%%%%%%%%%%%%%%%%%%%%%%%%%%%%%
    \usepackage[utf8]{inputenc} % Implementere Unicode
    \usepackage[T1]{fontenc}    % Unicode skrifttype, fx. é skrives som 1 tegn
   \usepackage[english]{babel} % Engelsk Ordbog
  % \usepackage[danish]{babel}  % Dansk Ordbog
    \usepackage{microtype}      % Forbedre linjeombrydningen
    \usepackage{libertine}      % Skrifttype
    \usepackage[scaled=0.83]{inconsolata} % Skrifttype til kode til kode
%%%%%%%%%%%%%%%%%%%%%%%%%%%%%%%%%%%%%%%%%%%%%%%%%%%%%%%%%%%%%%%%%%%%%%%%%%%%%%%%

% Pakker til matematik og kode.
%%%%%%%%%%%%%%%%%%%%%%%%%%%%%%%%%%%%%%%%%%%%%%%%%%%%%%%%%%%%%%%%%%%%%%%%%%%%%%%%
    \usepackage{mathtools}       % Udvidelse til amsmath pakken
   \usepackage{algpseudocode}   % pseudocode til algoritmer
    \usepackage{algorithm}       % Pakke til algoritmer
    \usepackage{amsthm}          % Pakke til Theroms
%%%%%%%%%%%%%%%%%%%%%%%%%%%%%%%%%%%%%%%%%%%%%%%%%%%%%%%%%%%%%%%%%%%%%%%%%%%%%%%%

% Pakker til layout.
%%%%%%%%%%%%%%%%%%%%%%%%%%%%%%%%%%%%%%%%%%%%%%%%%%%%%%%%%%%%%%%%%%%%%%%%%%%%%%%%
    \usepackage{fancyhdr}        % Gør det muligt at bruge sidehoveder
    \usepackage{graphicx}        % Mulighed for bl.a. \includegraphics
    \usepackage{parskip}         % Første paragraf i afsnit indrykkes ikke
    \usepackage{listings}        % Pakke til at indsætte kode
    \usepackage{enumitem}        % Gør det muligt at tilpasse lister
    \usepackage{titlesec}        % Tilpassing af afstand mellem sektioner
    \usepackage[lastpage,user]{zref} % Side x af y
%%%%%%%%%%%%%%%%%%%%%%%%%%%%%%%%%%%%%%%%%%%%%%%%%%%%%%%%%%%%%%%%%%%%%%%%%%%%%%%%
\usepackage{graphicx}
\usepackage{caption}
% Implementerer en række makroer og de pakker der er importeret
%%%%%%%%%%%%%%%%%%%%%%%%%%%%%%%%%%%%%%%%%%%%%%%%%%%%%%%%%%%%%%%%%%%%%%%%%%%%%%%%
    \pagestyle{fancy}                        % Implementerer sidehoved
    \lhead{} % Venstre sidehoved
    \rhead{Proactive Computer Security}      % Højre sidehoved
    \cfoot{\thepage\ of \zpageref{LastPage}} % Side x af y
    \newtheorem*{prp}{Propostion}            % Skaber nyt theorem
    \setlist{nolistsep}              % Formindsker mellemrum mellem listepunkter

    % Definitioner af farver
    %%%%%%%%%%%%%%%%%%%%%%%%%%%%%%%%%%%%%%%%%%%%%%%%%%%%%%%%%%%%%%%%%%%%%%%%%%%%
        \definecolor{KURed1}{RGB}{144,26,30}    % Official KU Red 1
        \definecolor{KURed2}{RGB}{199,36,41}    % Unofficial KU Red
        \definecolor{KUGray1}{RGB}{102,102,102} % Official KU Gray 1
        \definecolor{KUGray2}{RGB}{133,133,133} % Official KU Gray 2
        \definecolor{KUGray3}{RGB}{163,163,163} % Official KU Gray 3
        \definecolor{KUGray4}{RGB}{194,194,194} % Official KU Gray 4
        \definecolor{KUGray5}{RGB}{224,224,224} % Official KU Gray 5
    %%%%%%%%%%%%%%%%%%%%%%%%%%%%%%%%%%%%%%%%%%%%%%%%%%%%%%%%%%%%%%%%%%%%%%%%%%%%

    % Mindsker afstanden mellem sektioner
    %%%%%%%%%%%%%%%%%%%%%%%%%%%%%%%%%%%%%%%%%%%%%%%%%%%%%%%%%%%%%%%%%%%%%%%%%%%%
        \titlespacing\section{0pt}{12pt plus 4pt minus 2pt}
                                  {0pt plus 1pt minus 3pt}
        \titlespacing\subsection{0pt}{12pt plus 4pt minus 2pt}
                                  {0pt plus 1pt minus 3pt}
        \titlespacing\subsubsection{0pt}{12pt plus 4pt minus 2pt}
                                  {0pt plus 1pt minus 3pt}
    %%%%%%%%%%%%%%%%%%%%%%%%%%%%%%%%%%%%%%%%%%%%%%%%%%%%%%%%%%%%%%%%%%%%%%%%%%%%

    % Laver titel
    %%%%%%%%%%%%%%%%%%%%%%%%%%%%%%%%%%%%%%%%%%%%%%%%%%%%%%%%%%%%%%%%%%%%%%%%%%%%
    \title{
      \vspace{13em}
      \Large{Proactive Computer Security}\\
      \Huge{OP-table}
    }

    \author{
        \Large{Table master
            }
    }

    \date{
        \vspace{22em}
        \today \\
    }
    \lstset{language=Matlab}

%%%%%%%%%%%%%%%%%%%%%%%%%%%%%%%%%%%%%%%%%%%%%%%%%%%%%%%%%%%%%%%%%%%%%%%%%%%%%%%%
%%%%%%%%%%%%%%%%%%%%      Her starter dokumentet    %%%%%%%%%%%%%%%%%%%%%%%%%%%%
\begin{document}

\setcounter{page}{1}

    %% Change `ku-farve` to `nat-farve` to use SCIENCE's old colors or
    %% `natbio-farve` to use SCIENCE's new colors and logo.
    \AddToShipoutPicture*{\put(0,0){\includegraphics*[viewport=0 0 700 600]
    {include/ku-farve}}}
    \AddToShipoutPicture*{\put(0,602){\includegraphics*[viewport=0 600 700 1600]
    {include/natbio-farve}}}

    %% Change `ku-en` to `nat-en` to use the `Faculty of Science` header
    \AddToShipoutPicture*{\put(0,0){\includegraphics*{include/nat-en}}}
    \clearpage

%Disse linjer skaber forside, evt indholdsfortegnelse, og sætter sidetal
%%%%%%%%%%%%%%%%%%%%%%%%%%%%%%%%%%%%%%%%%%%%%%%%%%%%%%%%%%%%%%%%%%%%%%%%%%%%%
    \maketitle{}              % Forside
    \thispagestyle{empty}   % Fjerner sidetal forside
    \newpage                % Første rigtige side
%%%%%%%%%%%%%%%%%%%%%%%%%%%%%%%%%%%%%%%%%%%%%%%%%%%%%%%%%%%%%%%%%%%%%%%%%%%%%

\section{Interesting alphanumeric opcode}
\center
\begin{tabular}{l|l|}
    Char & Instruction\\
    \hline
    '0' & xor <r/m8>, <r8>\\
    '1' & xor <r/m32>, <r32>\\
    '2' & xor <r8>, <r/m8>\\
    '3' & xor <r32>, <r/m32>\\
    '4' & xor al, <imm8>\\
    '5' & xor eax, <imm32>\\
    '8' & cmp <r/m8>, <r8>\\
    '9' & cmp <r/m32>, <r32>\\
    'A' & inc ecx\\
    'B' & inc edx\\
    'C' & inc ebx\\
    'D' & inc esp\\
    'E' & inc ebp\\
    'F' & inc esi\\
    'G' & inc edi\\
    'H' & dec eax\\
    'I' & dec ecx\\
    'J' & dec edx\\
    'K' & dec ebx\\
    'L' & dec esp\\
    'M' & dec ebp\\
    'N' & dec esi\\
    'O' & dec edi\\
    'P' & push eax\\
\end{tabular}
\quad
\begin{tabular}{l|l}
    Char & Instruction\\
    \hline
    'Q' & push ecx\\
    'R' & push edx\\
    'S' & push ebx\\
    'T' & push esp\\
    'U' & push ebp\\
    'V' & push esi\\
    'W' & push edi\\
    'X' & pop eax\\
    'Y' & pop ecx\\
    'Z' & pop edx\\
    'a' & popa\\
    'h' & push <imm32>\\
    'j' & push <imm8>\\
    'p' & jo <disp8> \\
    'q' & jno <disp8>\\
    'r' & jb <disp8> \\
    's' & jae <disp8>\\
    't' & je <disp8> \\
    'u' & jne <disp8>\\
    'v' & jbe <disp8>\\
    'w' & ja <disp8> \\
    'x' & js <disp8> \\
    'y' & jns <disp8>\\
    'z' & jp <disp8> 
\end{tabular}
\end{document}
